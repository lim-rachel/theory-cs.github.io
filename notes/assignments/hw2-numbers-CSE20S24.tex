\input{../../resources/assignment-head.tex}

\title{hw2-numbers}
\date{Due: 4/16/24 at 5pm (no penalty late submission until 8am next morning)}
\begin{document}
\maketitle
\thispagestyle{fancy}


{\bf In this assignment,}

You will practice applying functions and tracing algorithms in multiple contexts, 
and exploring properties of positional number representations.

{\bf Relevant class material}: Week 2.

You will submit this assignment via Gradescope
(\href{https://www.gradescope.com}{https://www.gradescope.com}) 
in the assignment called ``hw2-numbers''.

\instructions

{\bf Assigned questions}

\begin{enumerate}[labelindent=0pt, leftmargin=0pt]
\item Functions and algorithms: in this question you'll explore the properties of the (integer) power and logarithm
functions. We'll use some definitions we introduced in class this week, namely the function
$b^i$ with domain $\mathbb{Z}^+ \times \mathbb{N}$ and codomain $\mathbb{N}$ defined recursively by
\begin{align*}    
&\textrm{Basis Step:} \\
&b^0 = 1 \\
&\textrm{Recursive Step:}\\
&\textrm{If } i \in \mathbb{N}, b^{i+1} = b \cdot b^i
\end{align*}
and the algorithm:
%! app: Numbers
%! outcome: number representation

\begin{algorithm}[caption={Calculating integer part of base $b$ logarithm}]
   procedure $logb$($b$,$n$: positive integers with $b > 1$)
   $i$ := $0$
   while $n$ > $b-1$
     $i$ := $i + 1$
     $n$ := $n$ div $b$
   return $i$ $\{ i$ holds the integer part of the base $b$ logarithm of $n\}$
\end{algorithm}


    \begin{enumerate}
    \item\gradeCorrectFirst Choose a positive integer $b$ between $3$ and $6$ (inclusive) and choose a 
    nonnegative integer $i$ between $2$ and $5$ (inclusive). Demonstrate how to calculate 
    the result of the $logb$ algorithm (procedure) when its input is the base $b$ you chose and
    $n = b^i$ (for the $i$ you choose.) 
    A complete answer will include the specific choice of $b$ and $i$, along with 
    trace of the calculations of $b^i$ and the computation of the algorithm, including 
    (clear, correct, complete) calculations for each of the function applications 
    and/or references to definitions and a trace table for algorithm that includes the values of all relevant 
    variables at each iteration (See the annotated Week 2 notes for the level of detail expected
    in a trace of a function application and a trace table).
    \item\gradeCorrect Now we'll go in other order: Demonstrate how to calculate the result of 
    $7^y$ where $y$ is the result of the $logb$ algorithm (procedure) when its input is $b=7$ and $n=30$.
    A complete answer will include clearly labelled 
    traces of the calculations of $b^i$ and the computation of the algorithm, including 
    (clear, correct, complete) calculations for each of the function applications 
    and/or references to definitions and a trace table for algorithm that includes the values of all relevant 
    variables at each iteration (See the annotated Week 2 notes for the level of detail expected
    in a trace of a function application and a trace table).    
    \item\gradeCompleteFirst Logarithms and powers are supposed to ``undo" one another. Explain whether your work in parts (a) and (b)
    supports that idea,  whether you saw anything confusing or surprising about these calculations, and how to explain what you saw.
    \end{enumerate}

\item Base expansions
    \begin{enumerate}
        \item\gradeComplete Pick an integer between $50$ and $1000$ (inclusive) that you (or one of your
        group members) came across at some point this week. 
        In a sentence or two, give some context for why you're choosing this number). 
        Write the base expansion of your chosen number in base $2$ (binary), base $3$ (ternary), base $4$, and base $16$ (hexadecimal).
        \item\gradeCorrect What is the {\bf smallest} width $w$ in which they could write your chosen number in base $8$ (octal) 
        fixed-width $w$? Justify your answer with reference to the definitions of fixed-width expansions and relevant calculations.
        \item\gradeCorrect Express in roster method the set of numbers between $1$ and $2$ (exclusive) that be written without error 
        (full precision)
        in binary fixed width expansion with integer part width $3$ and fractional part width $2$. Justify your answer with 
        reference to the definitions of fixed-width expansions and relevant calculations.
        \item\gradeCorrect Consider the strings of $1$s that have length $3$, $5$, $7$, and $10$.
        Calculate the numbers 
        \begin{itemize}
            \item[] $[111]_{s,3}$
            \item[] $[11111]_{s,5}$
            \item[] $[1111111]_{s,7}$
            \item[] $[1111111111]_{s,10}$
            \item[] $[111]_{2c,3}$
            \item[] $[11111]_{2c,5}$
            \item[] $[1111111]_{2c,7}$
            \item[] $[1111111111]_{2c,10}$
        \end{itemize}
        Justify your answers with specific reference to the definitions of sign-magnitude and $2$s complement expansions and relevant calculations.
        \item \gradeComplete What patterns do you notice in your calculations in part (d)?
    \end{enumerate}

\item Multiple representations

Recall that, mathematically, a color can be represented as a $3$-tuple $(r, g, b)$ 
where $r$ represents the red component, $g$ the green component, $b$ the blue component and where each of $r$, $g$, $b$ must be from the collection $\{x \in \mathbb{N}\mid 0 \leq x \leq 255 \}$.
As an alternative representation, in this assignment
we'll use base $b$ fixed-width expansions to represent colors
as individual numbers (this definition was introduced in this week's Review quiz).

{\bf Definition}: A {\bf hex color} is a nonnegative
integer, $n$, that has a base $16$ fixed-width $6$  expansion
$$n = (r_1r_2g_1g_2b_1b_2)_{16,6}$$ 
where $(r_1r_2)_{16,2}$ is the red
component, $(g_1g_2)_{16,2}$ is the green component, 
and $(b_1b_2)_{16,2}$ is the
blue.

For this question, let's call the set of hex colors $H$. In the Week 2 Review quiz (Question 3d), 
we explored a few different set builder definitions for $H$.

\begin{enumerate}
\item \gradeComplete Rewrite the set builder definition of a set below so that it 
refers to colors rather than numbers: $\{ c \in H \mid c \mod 256 = 0\}$. A complete answer will 
justify the new set builder definition by connection with the definition of hex colors and how it impacts
the colors that satisfy the specific property for this set.
\item \gradeCorrect Rewrite the set builder definition of a set below so that it 
refers to colors rather than numbers: $\{ c \in H \mid c < 65536\}$. A complete answer will 
justify the new set builder definition by connection with the definition of hex colors and how it impacts
the colors that satisfy the specific property for this set.
\item \gradeCorrect In art, we can mix two colors to get a new color. For example, red and blue make 
purple, blue and yellow make green, red and yellow make orange. 
A mathematical definition of {\bf hex color mixing} would be a function with domain $H \times H$
and codomain $H$ where the result of applying this function to an input $(n_1, n_2)$ is the hex color
that results from mixing the hex colors $n_1$ and $n_2$.


\rule{0.5\textwidth}{.4pt}

{\it Sample work for a related question that can be used as reference for the detail expected 
in your answer:} Consider the attempted mathematical definition  $f_1: H\times H \to H$ given by
\[
f_1 (~(n_1, n_2) ~) = n_1 + n_2
\]
We will show that this function is not well-defined so it cannot give a hex color mixing definition. 
The reason it is not well-defined
is that the application of the rule sometimes gives values that are not in the stated codomain. 
To see this, we need an example input to $f_1$ for which the output of the rule is not in $H$.
Here is one such example: $(n_1, n_2) = (~(FFFFFF)_{16,6}, (FFFFFF)_{16,6}~)$. 
This example is in $H \times H$
because $(FFFFFF)_{16,6}$ is a nonnegative integer that has a base $16$ fixed-width $6$ expansion 
so it is a hex color.
We now apply the definition of the rule in $f_1$:
\begin{align*}
f_1 (~(n_1, n_2) ~) &=  f_1 ( (~(FFFFFF)_{16,6}, (FFFFFF)_{16,6}~) ) = (FFFFFF)_{16,6} + (FFFFFF)_{16,6} \\
&= 2 \left( 15\cdot 16^5 + 15\cdot 16^4 + 15 \cdot 16^3 + 15 \cdot 16^2 + 15 \cdot 16^1 + 15 \cdot 16^0 \right) \\
&= 2 \cdot 15 \cdot (1048576 + 65536 + 4096 + 256 + 16 + 1) = 2 \cdot 15 \cdot 1118481 = 33554430
\end{align*}
This is not a hex color because it is greater than or equal to $16^6 = 16777216$, 
so (using the Week 2 notes on which numbers can be represented
with fixed-width expansions) it does not have a hexadecimal {\bf fixed width $6$} expansion.

\rule{0.5\textwidth}{.4pt}

In this question, we'll look at another attempted mathematical definition for hex color mixing. 
We define $f_2: H\times H \to H$ given by
\[
f_2 (~(n_1, n_2) ~) = (n_1 + n_2) \textbf{ div } 2
\]

This is a well-defined function. You do not need to prove this or hand it in, but it's good practice to make 
sure you understand why it's well-defined. Notice that the function application 
\[
f_2 ( ~(~ (FF0000)_{16,6} , (FFFE00)_{16,6} ~)~)
\]
can be calculated as:
\begin{align*}
f_2&(((FF0000)_{16,6}, (FFFE00)_{16,6}))=
 ((FF0000)_{16,6} + (FFFE00)_{16,6})) \textbf{ div } 2\\
&= \big(~ \left(15\cdot16^5 + 15\cdot16^4 + 0\cdot16^3 + 0\cdot16^2 + 0\cdot16^1 + 0\cdot16^0 \right) \\
&~~~~+~
    \left((15\cdot16^5 + 15\cdot16^4 + 15\cdot16^3 + 14\cdot16^2 + 0\cdot16^1 + 0\cdot16^0\right )~\big )\textbf{ div } 2\\
&= (16711680 + 16776704) \textbf{ div } 2 = 33488384 \textbf{ div } 2 = 16744192
\end{align*}
Since this is less than $16^6 = 16777216$, we can represent it using base $16$ fixed-width $6$:.
$$(16744192)_{10} = 15\cdot16^5 + 15\cdot16^4 + 7\cdot16^3 + 15\cdot16^2 + 0\cdot16^1 + 0\cdot16^0 = (FF7F00)_{16,6}$$

Using the web tool \url{https://www.w3schools.com/colors/colors_rgb.asp}, we can verify that $(FF0000)_{16,6}$ is red, 
$(FFFE00)_{16,6}$ is yellow, and $(FF7F00)_{16,6}$ is orange.

Show that $f_2$ does not work as a hex color mixing definition by finding
another ordered pair of hex colors for which the result of applying $f_2$ does not give the expected hex color.
Include the numerical description of each colors you mention, alongside a description of them in English. 
Describe how you know what these colors are
(if you use a web color tool, include its URL in your submission writeup; if not, describe your reasoning).
Justify your 
example with (clear, correct, complete) calculations and/or references to definitions, and connecting them with
the desired conclusion.
\end{enumerate}
\end{enumerate}
\end{document}