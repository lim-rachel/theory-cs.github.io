\input{../../resources/lesson-head.tex}
\section*{Let's get started}

We want you to be successful. 

We will work together to build an 
environment in CSE 20 that supports your learning
in a way that respects your
perspectives, experiences, and identities (including race, ethnicity, heritage, gender, sex, 
class, sexuality, religion, ability, age, educational background, etc.).
Our goal is for you to  engage
with interesting and challenging concepts and 
feel comfortable exploring, asking questions, and thriving.

If you or someone you know is suffering from food and/or housing insecurities 
there are UCSD resources here to help:

Basic Needs Office: \href{https://basicneeds.ucsd.edu/}{https://basicneeds.ucsd.edu/}

Triton Food Pantry (in the old Student Center)
is free and anonymous, and includes produce: 

\href{https://www.facebook.com/tritonfoodpantry/}{https://www.facebook.com/tritonfoodpantry/}

Mutual Aid UCSD: \href{https://mutualaiducsd.wordpress.com/}{https://mutualaiducsd.wordpress.com/}

Financial aid resources, the possibility of emergency grant funding, and off-campus housing referral 
resources are available: see your College Dean of Student Affairs.

If you find yourself in an uncomfortable situation, ask for help. 
We are committed to upholding University policies regarding nondiscrimination, sexual violence and sexual harassment.
Here are some campus contacts that could provide this help: 
Counseling and Psychological Services (CAPS) at 858 534-3755 or \href{http://caps.ucsd.edu}{http://caps.ucsd.edu}; 
OPHD at 858 534-8298 or ophd@ucsd.edu , \href{http://ophd.ucsd.edu}{http://ophd.ucsd.edu};
CARE at Sexual Assault Resource Center at 858 534-5793 or sarc@ucsd.edu , \href{http://care.ucsd.edu}{http://care.ucsd.edu}.


Please reach out (minnes@ucsd.edu) if you need support with extenuating circumstances affecting CSE 20.

\vfill

\section*{Welcome to CSE 20: Discrete Math for CS in Spring 2024!}
Class website: \href{https://canvas.ucsd.edu//}{https://canvas.ucsd.edu/}


Instructor: Prof. Mia Minnes {\tiny{"Minnes" rhymes with Guinness}}, minnes@ucsd.edu, 
\href{http://cseweb.ucsd.edu/~minnes}{http://cseweb.ucsd.edu/~minnes}


Our team: One instructor + two TAs and eleven tutors + all of you

Fill in contact info for students around you, if you'd like:

\vfill


On a typical week in CSE 20: {\bf MWF} Lectures (sometimes with pre-class reading), {\bf W} Discussion,
Review quiz, then {\bf T} Homework due.
Office hours (hosted by instructors and TAs and tutors) where you can come to talk 
about course concepts and ask for help as you work through sample problems 
and Q+A on Piazza available throughout the week. CSE 20 has one project and two tests
this quarter.
Demonstration of class website on \href{https://canvas.ucsd.edu/}{Canvas (https://canvas.ucsd.edu/)}:
\begin{enumerate}
\item Syllabus
\item Notes for class + annotations
\item Assignments (PDF, tex, solutions)
\item Gradescope
\item Piazza
\item Dates
\end{enumerate}

\vfill
\begin{comment}
There are lots of great reasons to have a laptop, tablet, or phone open during class. You might be taking notes, 
getting a photo of an important moment on the board, trying out a construction that we're developing together, working 
on the review quiz, and so on. 
The main issue with screens and technology in the classroom isn't that they might 
distract you, 
it's the distraction of other students. We ask that if you would like to use
a device in class and may have 
have unrelated content open, please sit in one of the back two rows of the 
room so that it's not adversely affecting other students.


\vfill
\end{comment}
{\bf Pro-tip}: you can use MATH109 to replace CSE20 for prerequisites and other requirements.

\vfill

\section*{Themes and applications for CSE 20}
\begin{itemize}
\item {\bf Technical skepticism}: Know, select and apply appropriate computing knowledge and problem-solving techniques. 
Reason about computation and systems. 
Use mathematical techniques to solve problems. 
Determine appropriate conceptual tools to apply to new situations. 
Know when tools do not apply and try different approaches. 
Critically analyze and evaluate candidate solutions.
\item {\bf Multiple representations}: Understand, guide, shape impact of computing on society/the world. 
Connect the role of Theory CS classes to other applications (in undergraduate CS curriculum and beyond). 
Model problems using appropriate mathematical concepts.
Clearly and unambiguously communicate computational ideas using appropriate formalism. 
Translate across levels of abstraction.
\end{itemize}

{\bf Applications}: Numbers (how to represent them and use them in Computer Science), 
Recommendation systems and their roots in machine learning (with applications like Netflix),
``Under the hood" of computers (circuits, pixel color representation, data structures),
Codes and information (secret message sharing and error correction),
Bioinformatics algorithms and genomics (DNA and RNA).

\newpage

\subsection*{Week 1 at a glance}

\subsubsection*{We will be learning and practicing to:}
%data types, translating, important sets, write set definition, function and relation definitions, recursive definitions
\begin{itemize}
\item Model systems with tools from discrete mathematics and reason about implications of modelling choices. Explore applications in CS through multiple perspectives, including software, hardware, and theory.
\begin{itemize}

   \item Selecting and representing appropriate data types and using notation conventions to clearly communicate choices

\end{itemize}

\item Translate between different representations to illustrate a concept.

\begin{itemize}
   \item Translating between symbolic and English versions of statements using precise mathematical language
\end{itemize}


\item Use precise notation to encode meaning and present arguments concisely and clearly
\begin{itemize}
    \item Defining important sets of numbers, e.g. set of integers, set of rational numbers
    \item Precisely describing a set using appropriate notation e.g. roster method, set builder notation, and recursive definitions
    \item Defining functions using multiple representations
\end{itemize}

\item Know, select and apply appropriate computing knowledge and problem-solving techniques. Reason about computation and systems. Use mathematical techniques to solve problems. Determine appropriate conceptual tools to apply to new situations. Know when tools do not apply and try different approaches. Critically analyze and evaluate candidate solutions.
\begin{itemize}
    \item Using a recursive definition to evaluate a function or determine membership in a set
\end{itemize}

\end{itemize}

\subsubsection*{TODO:}
\begin{list}
   {\itemsep2pt}
   \item \#FinAid Assignment on Canvas (complete as soon as possible) 
   \item Review quiz based on class material each day (due Friday April 5, 2024)
   \item Homework assignment 1 (due Tuesday April 9, 2024).
\end{list}

\newpage

\section*{Week 1 Monday: Modeling applications}
\input{../activity-snippets/netflix-intro.tex}

\vfill
\subsection*{Data Types: sets, $n$-tuples, and strings}
\input{../activity-snippets/data-types.tex}
\newpage
%! app: Recommendation Systems
%! outcome: data types

In the table  below,  each row represents a user's ratings of movies: 
\cmark~(check) indicates the person liked the movie, \xmark~(x)
that they didn't, and $\bullet$ (dot) that they didn't rate it one way or 
another (neutral rating or didn't watch). Can encode
these ratings numerically with $1$ for \cmark~(check), $-1$ for \xmark~(x), 
and $0$ for $\bullet$ (dot).

\vfill

\begin{center}
\begin{tabular}{c|cccc||c}
Person & Dune & Oppenheimer & Barbie & Nimona & Ratings written as a $4$-tuple\\
\hline
$P_1$     & \xmark & $\bullet$ & \cmark & \phantom{$(-1, 0, 1, 1)$} \\
&&&& \\
$P_2$     & \cmark & \cmark & \xmark & \phantom{$(1, 1, -1, 1)$} \\
&&&& \\
$P_3$     & \cmark & \cmark & \cmark & \phantom{$(1, 1, 1, 1)$} \\
&&&& \\
$P_4$     & $\bullet$ & \xmark & \cmark &  \\
&&&& \\
$You$     &  &  &  &  \\
&&&& \\
\end{tabular}
\end{center}
\vfill
{\bf Conclusion}: Modeling involves choosing data types to represent and organize data

\newpage
\section*{Week 1 Wednesday: Defining sets}
\subsection*{Notation and prerequisites}
\input{../activity-snippets/definitions-set-prereqs.tex}
%! app: Numbers, Recommendation Systems, Bioinformatics
%! outcome: data types, write set definition, important sets

{\it To define sets:}

To define a set using {\bf roster method}, explicitly list its elements. That is,
start with $\{$ then list elements of 
the set separated by commas and close with $\}$.

\vfill

To define a set using {\bf set builder definition}, either form 
``The set of all $x$ from the universe $U$ such that $x$ is ..." by writing
\[\{x \in U \mid ...x... \}\]
or form ``the collection of all outputs of some operation when the input ranges over the universe $U$"
by writing
\[\{ ...x... \mid x\in U \}\]

\vfill

We use the symbol $\in$ as ``is an element of'' to indicate membership in a set.\\

\newpage 

{\bf Example sets}: For each of the following, identify whether it's defined using the roster method
or set builder notation and give an example element.

Can we infer the data type of the example element from the notation?

\begin{itemize}
    \item[]$\{ -1, 1\}$
    \vfill
    \item[]$\{0, 0 \}$
    \vfill
    \item[]$\{-1, 0, 1 \}$
    \vfill
    \item[]$\{(x,x,x) \mid x \in \{-1,0,1\} \}$
    \vfill
    \item[]$\{ \}$
    \vfill
    \item[]$\{ x \in \mathbb{Z} \mid x \geq 0 \}$
    \vfill
    \item[]$\{ x \in \mathbb{Z}  \mid x > 0 \}$
    \vfill
    \item[]$\{ \smile, \sun \}$
    \vfill
    \item[]$\{\A,\C,\U,\G\}$
    \vfill
    \item[]$\{\A\U\G, \U\A\G, \U\G\A, \U\A\A \}$
    \vfill
\end{itemize}

\newpage
\input{../activity-snippets/rna-motivation.tex}
\input{../activity-snippets/recursive-sets-definition.tex}
\input{../activity-snippets/set-recursive-examples.tex}
\input{../activity-snippets/set-operations.tex}

\section*{Week 1 Friday: Defining functions}
\input{../activity-snippets/definitions-functions-prereqs.tex}
%! app: TODOapp
%! outcome: function and relation definitions, data types

\fbox{\parbox{\textwidth}{%
{\bf New! Defining functions} A function is defined by its (1) domain, 
(2) codomain, and (3) rule assigning each 
element in the domain exactly one element in the codomain.\\

The domain and codomain are nonempty sets.

The rule can be depicted as a table, formula, piecewise definition, or English description.

The notation is 
\begin{center}
    ``Let the function FUNCTION-NAME: DOMAIN $\to$ CODOMAIN be given by \\
FUNCTION-NAME(x) = \ldots for every $x \in DOMAIN$''.
\end{center}

or 
\begin{center}
    ``Consider the function FUNCTION-NAME: DOMAIN $\to$ CODOMAIN defined as  \\
FUNCTION-NAME(x) = \ldots for every $x \in DOMAIN$''.
\end{center}
}}

\vfill
\newpage

Example: The absolute value function 

{\bf Domain}

{\bf Codomain}

{\bf Rule}

\vfill 

\input{../activity-snippets/defining-functions-ratings.tex}
\input{../activity-snippets/defining-functions-recursively.tex}

\newpage
\subsection*{Review Quiz}
\begin{enumerate}
\item Modeling
\begin{enumerate}
    \item {%! app: Recommendation Systems
%! outcome: data types

Using the example movie database from class with the four movies Dune, Oppenheimer, Barbie, and Nimona which of the following is a $4$-tuple that represents the ratings of a user who liked Dune? (Select all and only that apply.)

\begin{enumerate}
\item $1$
\item $(1,0,0)$
\item $[1,1,1,0]$
\item $\{-1, 0, 0, 1\}$
\item $(1,-1,0,1)$
\item $(0,1,1, 1)$
\item $(1,1,1,1)$
\end{enumerate}}
    \item {%! app: Recommendation Systems
%! outcome: data types

Using the example movie database from class with the four movies Dune, Oppenheimer, Barbie, and Nimona how many distinct (different) $4$-tuples of ratings are there? 
}
    \item {\input{../activity-snippets/quiz-color-rgb-definitions.tex}}
    \item {%! app: Computers
%! outcome: data types, write set definition, important sets


In the definition of colors as amounts of red, green, and blue mixed together, why are $3$-tuples a convenient data structure to use rather than sets or strings?

(Select all and only relevant choices)

\begin{enumerate}
\item Ordering matters in $n$-tuples, so we can use the different components of the $3$-tuple to represent the amounts of specific colors. 
\item There are many possible values for each color amount and we don't have individual characters for each value so a string could get unwieldy.
\item It's possible to have the same value of two or all of the colors, and repetition matters in $n$-tuples.
\end{enumerate}
\vfill}
\end{enumerate}
\item Sets and functions
\begin{enumerate}
    \item {\input{../activity-snippets/quiz-set-membership.tex}}
    \item {\input{../activity-snippets/quiz-strands-set-operations.tex}}
    \item {%! app: Recommendation Systems
%! outcome: function and relation definitions, data types

Recall the function which takes an ordered pair of ratings $4$-tuples and returns a measure of the difference between them
$d_0: \{-1,0,1\}^4 \times \{-1,0,1\}^4 \to \mathbb{R}$
given by
\[
d_0 (~(~ (x_1, x_2, x_3, x_4), (y_1, y_2, y_3, y_4) ~) ~) = \sqrt{ (x_1 - y_1)^2 + (x_2 - y_2)^2 + (x_3 -y_3)^2 + (x_4 -y_4)^2}
\]

Consider the function application 
\[
  d_0 (~( ~(-1,1,1, 0), (1, 0, -1, 0)~) ~)
\]
\begin{enumerate}
    \item What is the input? 
    \item What is the output?
\end{enumerate}}
%    \item {\input{../activity-snippets/quiz-recursive-definitions.tex}}
%    \item {\input{../activity-snippets/quiz-defining-functions-recursively.tex}}
\end{enumerate}

\end{enumerate}

\end{document}